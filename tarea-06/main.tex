\documentclass{article}
%
% Demo of the mcode package from 
% http://www.mathworks.co.uk/matlabcentral/fileexchange/8015-m-code-latex-package
% Updated 06 Mar 2014
%

% load package with ``framed'' and ``numbered'' option.
\usepackage[framed,numbered,autolinebreaks,useliterate]{mcode}
\usepackage[spanish]{babel}
\usepackage{graphicx}
\graphicspath{ {images/} }

\usepackage{listings}
\usepackage{xcolor}

\lstset{
    language=lisp,
    basicstyle=\ttfamily\small,
    aboveskip={1.0\baselineskip},
    belowskip={1.0\baselineskip},
    columns=fixed,
    extendedchars=true,
    breaklines=true,
    tabsize=4,
    prebreak=\raisebox{0ex}[0ex][0ex]{\ensuremath{\hookleftarrow}},
    frame=lines,
    showtabs=false,
    showspaces=false,
    showstringspaces=false,
    keywordstyle=\color[rgb]{0.627,0.126,0.941},
    commentstyle=\color[rgb]{0.133,0.545,0.133},
    stringstyle=\color[rgb]{01,0,0},
    numbers=left,
    numberstyle=\small,
    stepnumber=1,
    numbersep=10pt,
    captionpos=t,
    escapeinside={\%*}{*)}
}

\lstset{style=mystyle}

% something NOT relevant to the usage of the package.
\usepackage{url}
\setlength{\parindent}{0pt}
\setlength{\parskip}{18pt}
\title{ Problemas EXPLICIT-REFS }
\author{Ricardo Payan}
% //////////////////////////////////////////////////

\begin{document}

\maketitle

\section*{Ejercicio 4.8.}


Show exactly where in our implementation of the store these operations take linear time rather than constant time.



- En newref al usar length y append hacemos que la funcion trabaje con tiempo lineal.

- En deref, list-ref tiene tiempo de ejecucion lineal

- En setref!, setref-inner funciona con un ciclo, asi que setref! tiene tiempo de ejecucion lineal.


\section*{Ejercicio 4.9}
Implement the store in constant time by representing it as a Scheme vector. What is lost by using this representation?

\begin{lstlisting}
(define (empty-store)
	(vector))

(define the-store 'uninitialized)

(define (get-store)
  the-store)

(define reference?
	(lambda (v)
		(integer? v)))

(define (extend-store store val)
	(let* ([store-size (vector-length store)]
				[new-store (make-vector (+ store-size 1) (vector-copy! store 0 store 0 (vector-length store)))])
	(vector-set! new-store (vector-length (new-store) val))))

(define newref val
	(let* ([new-store-info (extend-store the-store val)]
					[new-store (first new-store-info)]
					[next-ref (last new-store-info)])
		(set! the-store new-store) 
			next-ref))

(define (deref ref)
(vector-ref the-store ref))

(define (setref! ref val)
	(cond
	[(and (reference? ref) (< ref (vector-length the-store))) (vector-set! the-store ref val)]
	[error 'setref! "No se puede cambiar la referencia"]))
\end{lstlisting}


\section*{Ejercicio 4.10}
Implement the begin expression as specified in exercise 4.4.

\begin{lstlisting}
#Sintaxis Concreta
    Expression := begin Expression {; Expresion }* end

#Sintaxis Abstracta
    begin-exp(exps)
\end{lstlisting}

\begin{lstlisting}
#Implementacion
(value-of (begin-exp (exp1 (cons exp2 null))) env) =
	((value-of exp1 env)
		(pair-val (cons (value-of exp2 env)
										(null-val))))
\end{lstlisting}

\section*{Ejercicio 4.11}
Implement list from exercise 4.5.



\begin{lstlisting}
#Sintaxis concreta 
    Expression := list (Expression *(,))

#Sintaxis Abstracta
    list-exp(exps)
\end{lstlisting}

\begin{lstlisting}
#Implementacion

(value-of (list-exp (cons exp1 (cons exp2 null))) env) =
	(pair-val (cons (value-of exp1 env)
			(pair-val(cons (value-of exp2 env)
											(null-val)))))
\end{lstlisting}


\end{document}
